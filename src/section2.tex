
\section{Problem Settings}
\label{sec:problem-settings}
%
In this section, we formulate a distributionally robust coreset selection (DRCS) problem in the context of uncertain covariate-shift settings.

\subsection{Preliminaries and Notations}
%
We consider a binary classification problem with the training dataset ${\mathcal D} := \{(\bm x_i, y_i)\}_{i \in [n]}$, where $n$ denotes the number of training instances, $\bm x_i \in \cX \subseteq {\mathbb R}^d$ represents the $i$-th feature vector defined on the input domain $\cX$, and $y_i \in \{\pm 1\}$ denotes the corresponding label for $i \in [n]$, with the notation $[n]$ indicating the set of natural numbers up to $n$.  
%
Similarly, let ${\mathcal D}^\prime := \{(\bm x_i^\prime, y_i^\prime)\}_{i \in [n^\prime]}$ be the validation dataset of size $n^\prime$, where each validation instance is assumed to follow the same distribution as the training instances.
%
For binary classification problems, we consider a classifier parameterized by a set of parameters $\bm \beta$, defined as  
\begin{align}  
\label{eq:classifier}  
f(\cdot; \bm \beta): \mathbb{R}^d \ni \bm x \mapsto f(\bm x; \bm \beta) \in \mathbb{R},  
\end{align}  
where the binary label for an input vector $\bm x$ is predicted as $-1$ if $f(\bm x; \bm \beta) < 0$, and $+1$ otherwise.  
%
Furthermore, we denote the loss function for binary classification (e.g., binary cross-entropy or hinge loss) as
\begin{align}
 \label{eq:loss_func}
 \ell: \{\pm 1\} \times \RR \ni (y, f(\bm x; \bm \beta)) \mapsto \ell(y, f(\bm x; \bm \beta)) \in \RR_+,
\end{align}
where $\RR_+$ indicates the set of nonnegative numbers~\footnote{
%
For example, in the case of the binary cross-entropy loss for a logistic regression model, it is expressed as  
\begin{align*}  
\ell(y, f(\bm x; \bm \beta)) = \log (1 + e^{-y f(\bm x; \bm \beta)}).  
\end{align*}  
%
As another example, the hinge loss for a support vector machine (SVM) is expressed as  
\begin{align*}  
\ell(y, f(\bm x; \bm \beta)) = \max \{0, 1 - y f(\bm x; \bm \beta)\}.  
\end{align*}  
}.

In this study, we investigate distributionally robust learning under an uncertain covariate shift setting, where the input distributions for the training/validation datasets and the test datasets differ, with the discrepancy between these distributions known to lie within a specified range.
%
In such a covariate-shift setting, it is well known that the difference in input distributions can be addressed through weighted learning.  
%
We denote the weight for the $i$-th training instance as $w_i > 0, i \in [n]$, and represent the $n$-dimensional vector of these weights as $\bm{w} \in [0, \infty)^n$.  
%
Typically, these weights are determined based on the ratio of the test input density to the training/validation input density~\citep{shimodaira2000improving, sugiyama2007covariate}.  
%
Therefore, a binary classification problem in a covariate-shift setting is generally formulated as a (regularized) weighted empirical risk minimization problem in the form of  
\begin{align}  
\label{eq:weighted_minimization}  
\min_{\bm \beta} {\frac{1}{\sum_{i \in [n]} w_i}} \sum_{i \in [n]} w_i \ell(y_i, f(\bm x_i; \bm \beta)) + \rho(\bm \beta),  
\end{align}  
where $\rho(\bm \beta)$ denotes a regularization function.  

In a conventional covariate-shift setting where the test input distribution is known, the weight vector $\bm{w}$ can be predetermined based on the density ratio, enabling the optimal model parameters to be obtained by directly solving the minimization problem in \eqref{eq:weighted_minimization}.
%
On the other hand, in the distributionally robust setting, the density ratio and hence the weight vector $\bm w$ are unknown.
%
In this study, we consider a distributionally robust learning scenario where the weight vector lies within a hypersphere of a certain radius $S$, centered around the uniform training data weights (i.e., $ \bm w = \bm{1}_n := [1, \ldots, 1]^\top $).
%
This distributionally robust learning problem is formulated as  
\begin{align}  
\red{\min_{\bm{\beta}} \max_{\bm{w} \in \cW}} {\frac{1}{\sum_{i \in [n]} w_i}} \sum_{i \in [n]} w_i \ell(y_i, f(\bm{x}_i; \bm{\beta})) + \rho(\bm{\beta}),  
\end{align}  
where  
\begin{align}  
\cW := \{\bm{w} \in \mathbb{R}^n \mid \| \bm{w} - \bm{1}_{n} \|_2 \le S\} \label{eq:weight-changes-distrib}
\end{align}  
represents the hypersphere within which the weight vector can exist~\footnote{  
We assume $S \leq 1$, focusing on scenarios where the differences between the training and test input distributions are not substantial.  
}.  


\subsection{Applicable Class of Problems by the Proposed Method} \label{sec:applicable-class}
%
Before presenting the proposed method, we define the class of classification problems to which the DRCS method applies.
%
The proposed DRCS method is applicable when the classifier in \eqref{eq:classifier}, the loss function in \eqref{eq:loss_func}, and the regularization function in \eqref{eq:weighted_minimization} satisfy certain conditions.
%
As a class of applicable classifiers, we focus on the basis function model in the form of
\begin{align}
 \label{eq:basis_expansion}
 f(\bm{x}; \bm{\beta}) = \sum_{j\in [k]} \beta_j \phi_j(\bm{x}) = \bm{\beta}^\top \bm{\phi}(\bm{x}),
\end{align}

where $\phi_j: \mathbb{R}^d \ni \bm{x} \mapsto \phi_j(\bm{x}) \in \mathbb{R}, j \in [k]$ is the $j$-th basis function, $k$ is the number of basis functions, and $\bm{\phi}(\bm{x}) \in \mathbb{R}^k$ is the $k$-dimensional vector that gathers the $k$ basis functions.
%
Furthermore, the loss function $\ell(y, f(\bm{x}; \bm{\beta}))$ is assumed to be convex with respect to its second argument, and the regularization function $\rho(\bm{\beta})$ must also be convex with respect to $\bm{\beta}$.

While these conditions may seem restrictive, kernelization is achievable by considering the dual form of the basis function model in \eqref{eq:basis_expansion}, enabling extensions to popular nonlinear classifiers such as nonlinear logistic regression and kernel support vector machines.  
%
When applying the proposed method to deep learning models, tools such as Neural Tangent Kernel (NTK)~\citep{neuraltangents2020} and Neural Network Gaussian Processes (NNGP)~\citep{lee2017deep} can be utilized.
%
These tools bridge traditional kernel methods with deep learning, offering both theoretical insights and practical tools for tackling high-dimensional and complex learning problems.  
%
Furthermore, in practical DRCS problems, since the training and test data are typically assumed to share a certain degree of similarity, a fine-tuning approach that updates part of the model parameters during testing is beneficial.  
%
The proposed method is applicable in such scenarios and serves as a valuable tool for coreset selection in deep learning models as well.
% %
% See Section~\ref{xxx} for more details on applying the proposed method to deep learning models.








% ========= 2.3 =========

\subsection{Distributionally Robust Coreset Selection (DRCS) Problems}
%
To formulate the coreset selection problem, let us introduce an $ n $-dimensional binary vector $ \bm v \in \{0, 1\}^n $, where $ v_i = 1 $ indicates that the $ i $-th training instance is included in the coreset, while $ v_i = 0 $ indicates that the $ i $-th training instance is excluded.  
%
Hereafter, we refer to $ \bm v $ as the \emph{coreset vector}.  
%
Let us write the training error of our DRCS problem as
\begin{align}
\label{eq:primal}
P_{\bm v, \bm w}(\bm \beta)
:=
\frac{1}{\sum_{i \in [n]} v_i w_i}
\sum_{i \in [n]}
v_i
w_i
\ell
\left(
y_i, f(\bm x_i; \bm \beta)
\right)
+
\rho(\bm \beta). 
\end{align}
%
Given a coreset vector $\bm v \in \{0, 1\}^n$ and a weight vector $\bm w \in \cW$, the classifier's optimal model parameter vector is written as
\begin{align}
 \label{eq:beta-star}
 \bm \beta^*(\bm v, \bm w)
 :=
 \argmin_{\bm \beta}
 P_{\bm v, \bm w}(\bm \beta).
\end{align}

Given a weight vector $\bm{w}^\prime \in \mathbb{R}^{n^\prime}$ for validation dataset, the weighted validation error for the optimal model parameter vector in \eqref{eq:beta-star} can be defined as  
\begin{align}
\label{eq:VaEr}
{\rm VaEr}(\bm{v}, \red{\bm{w}, \bm{w}^\prime}) 
:= 
\frac{1}{\sum_{i \in [n^\prime]} w_i^\prime} 
\sum_{i \in [n^\prime]} 
w_i^\prime 
I 
\left\{ 
y_i^\prime \neq {\rm sgn}(f(\bm{x}_i^\prime; \bm{\beta}^*(\bm{v}, \bm{w}))) 
\right\},
\end{align}
where  
\begin{align}  
\cW^\prime := \{\bm{w}^\prime \in \mathbb{R}^{n^\prime} \mid \| \bm{w}^\prime - \bm{1}_{n^\prime} \|_2 \le Q\}  
\end{align}  
represents the hypersphere of a certain radius $Q$ within which the weight vector can exist.
%
Here, $I\{\cdot\}$ is the indicator function that returns $1$ if the argument is true and $0$ otherwise, and ${\rm sgn}: \RR \to \{\pm 1\}$ is the sign function that returns the sign of the given scalar input. 
%
If the weight vector $\bm w^\prime$ for the validation dataset is appropriately determined based on the density ratio of the input distributions of the validation and the test datasets, the weighted validation error ${\rm VaEr}$ in \eqref{eq:VaEr} can be used as a test error estimator.

In a distributionally robust setting where the weight vector is uncertain, minimizing the worst-case test error is necessary.  
%
As an estimator of the worst-case test error, it is reasonable to use the worst-case weighted validation error (WrVaEr) in \eqref{eq:VaEr}, expressed as  
\begin{align}
\label{eq:WrVaEr}
{\rm WrVaEr}(\bm{v}) 
= 
\red{\max_{\bm{w} \in {\cW}^\prime}
\max_{\bm{w} \in {\cW}}}
{\rm VaEr}(\bm{v}, \red{\bm{w}, \bm{w}^\prime}).
\end{align}
%
Based on the above discussion, the goal of the DRCS problem is formulated as the problem of finding the coreset vector $\bm v \in \{0, 1\}^n$ that minimizes the worst-case weighted validation error in \eqref{eq:WrVaEr}.
%
Let
$m < n$
be the size of the coreset, i.e., the number of remaining training instances after coreset selection.
%
Then, the DRCS problem we consider is formulated as 
\begin{align}
\label{eq:DRCS-problem}
\bm{v}^* = \argmin_{\bm{v}} {\rm WrVaEr}(\bm{v})
 ~
\text{ subject to } \|\bm{v}\|_1 = m.
\end{align}

By summarizing equations \ref{eq:beta-star}, \ref{eq:VaEr}, \ref{eq:WrVaEr} and \ref{eq:DRCS-problem}, minimizing the worst-case weighted validation error (WrVaEr) can be expressed as:
\begin{equation} \label{eq:obj-acc-lb}
  \min_{\bm v} \red{\max_{\bm w^\prime}} \max_{\bm w} \min_{\bm \beta} {\rm VaEr}(\bm{v}, \bm{w}, \red{\bm{w}^\prime}, \bm{\beta}).
\end{equation}

Unfortunately, the problem in \eqref{eq:DRCS-problem} is highly challenging.  
%
\red{First, it involves complex optimization over four types of vectors: $ \bm{v} $, $ \bm{w} $, $ \bm{w}^\prime $, and $ \bm{\beta} $.}
%
This optimization is technically complex as it requires solving nested optimization problems across \red{four} hierarchical levels, involving a large number of variables and significantly increasing computational complexity, along with the difficulty of ensuring convergence and global optimality.  
%
Moreover, the \red{fourth}-level optimization for $ \bm{v} \in \{0, 1\}^n $ is a combinatorial optimization problem, which becomes infeasible to solve optimally when $ n $ is large.  
%
To address these technical challenges, our approach involves the following two steps:  
\begin{itemize}  
 \item [(i)] Deriving an upper bound for the worst-case weighted validation error in \eqref{eq:WrVaEr}.  
 \item [(ii)] Greedily identifying a coreset that minimizes this upper bound of worst-case weighted validation error.
\end{itemize}
%
In the next section, we describe methods for (i) and (ii) in Section~\ref{subsec:UB_WC} and Section~\ref{subsec:GrCS}, respectively.  


% ========= 2.4 =========
\subsection{A Working Example: $L_2$-regularized Logistic Regression}
%
As a working example of such a classification problem, let us consider $L_2$-regularized logistic regression of basis function models, i.e., we consider a basis function model in \eqref{eq:basis_expansion}, binary cross-entropy as the loss function $\ell(y, f(\bm x; \bm \beta))$ and $L_2$ regularization function as the regularization function $\rho(\bm \beta)$.
%
Considering the coreset vector $\bm v$, the optimization problem for $L_2$-regularized logistic regression is defined as  
\begin{align}
\bm \beta_{\bm v, \bm w}^* = \argmin_{\bm \beta \in \mathbb{R}^k} P_{\bm v, \bm w}(\bm \beta),
\end{align}
where  
\begin{align}
% \label{eq:primal}
P_{\bm v, \bm w}(\bm \beta)
=
\frac{1}{\sum_{i \in [n]} v_i w_i}
\sum_{i \in [n]}
v_i w_i
\log (1 + e^{-y_i f(\bm x_i; \bm \beta)})
+
\frac{\lambda}{2} \|\bm \beta\|_2^2, 
\end{align}
which is referred to as the \emph{primal problem} and $P_{\bm v, \bm w}(\bm \beta)$ is called \emph{primal objective function}.
