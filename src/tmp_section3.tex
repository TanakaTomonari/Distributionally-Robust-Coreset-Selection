% ------------------------------------------------------------------------------------------
% sec3
% ------------------------------------------------------------------------------------------
\section{Proposed DRCS Method}
\label{sec:method}
%
In this section, we present the proposed DRCS method that approximately solve the DRCS problem in \eqref{eq:DRCS-problem}.
%
As discussed in the previous section, our approach to approximately finding the optimal coreset vector $\bm{v}^*$ involves deriving an upper bound for \eqref{eq:main-bound} and selecting the coreset by greedily minimizing this upper bound.



\subsection{Upper Bound for the Worst-case Test Error}
\label{subsec:UB_WC}
%
In this subsection, we present our main result.

\subsubsection{Main Theorem}
%
An upper bound of the worst-case weighted validation error, which can serve as the worst-case test error estimator, $ {\rm WrVaEr}(\bm v) $ in \eqref{eq:WrVaEr}, is presented in the following theorem. Our main result can be formulated as \eqref{eq:main-bound}.
%
\begin{theorem}
 \label{theo:main}
 {Assume that the primal objective function is $\mu$-strongly convex function with respect to $\bm \beta$.}
 %
 Let us denote the optimal primal and dual solutions for the entire training set (i.e., $v_i = 1 ~ \forall i \in [n]$) with uniform weights (i.e., $w_i = 1 ~ \forall i \in [n]$) as
 \begin{align*}
  \bm \beta^*_{\bm 1_n, \bm 1_n} = \argmin_{\bm \beta \in \RR^k} P_{\bm 1_n, \bm 1_n}(\bm \beta)
 \end{align*}
 and 
 \begin{align*}
  \bm \alpha^*_{\bm 1_n, \bm 1_n} = \argmin_{\bm \alpha \in \RR^n} D_{\bm 1_n, \bm 1_n}(\bm \alpha),
 \end{align*}
 respectively.
 %
 Then, an upper bound of the worst-case weighted validation error is written as 
 \begin{align}
  \label{eq:main-bound}
  {\rm WrVaEr}(\bm v)
  \le
  {\rm WrVaEr}^{\rm UB}(\bm v)
  =
  1
  -
  \left(
  \bm 1_{n^\prime}^\top \bm \zeta(\bm v)
  -
  Q
  \sqrt{
  \|\bm \zeta(\bm v)\|_2^2
  -
  \frac{
  (\bm 1_{n^\prime}^\top \bm \zeta(\bm v))^2
  }{
  n^\prime
  }
  }
  \right)
  \frac{1}{n^\prime},
 \end{align}
 where
 $\bm \zeta(\bm v) \in \{0, 1\}^{n^\prime}$
 is an $n^\prime$-dimensional binary vector whose $i$-th element is defined as
 \begin{align}
  \label{eq:zeta}
  \zeta_i(\bm v)
  =
  I\left\{
  y_i^\prime \bm \beta_{\bm 1_n, \bm 1_n}^{*\top} \bm \phi(\bm x_i^\prime) - \|\bm \phi(\bm x_i^\prime)\|_2 \sqrt{\frac{2}{\lambda} \max_{\bm w \in \cW} {\rm DG}(\bm v, \bm w)} > 0
  \right\}.
 \end{align}
 %
 Here, ${\rm DG}$ is the duality gap for the problem with the coreset $\bm v$ and the weights $\bm w$ with respect to the primal and dual solutions $\bm \beta^*_{\bm 1_n, \bm 1_n}$ and $\bm \alpha^*_{\bm 1_n, \bm 1_n}$ for the training set, which is expressed as  
\begin{align}
 \label{eq:duality_gap}
{\rm DG}(\bm v, \bm w) := P_{\bm v, \bm w}(\bm \beta^*_{\bm 1_n, \bm 1_n}) - D_{\bm v, \bm w}(\hat{\bm \alpha}_{\bm 1_n, \bm 1_n}^{*}),
\end{align}
where  
\begin{align*}
  \hat{\bm \alpha}_{\bm 1_n, \bm 1_n}^{*} = \{\bm \alpha^*_{\bm 1_n, \bm 1_n}\}_{v_i = 1}  
\end{align*}
is the $|\bm v|$-dimensional subvector extracted from the $n$-dimensional vector $\bm \alpha^*_{\bm 1_n, \bm 1_n}$, containing only the elements corresponding to indices where $v_i = 1$.
%
Furthermore, $ \max_{\bm w \in \cW} {\rm DG}(\bm v, \bm w) $  
in \eqref{eq:zeta} can be analytically obtained by solving an eigenvalue problem that maximizes the following convex quadratic form:  
\begin{align}  
\label{eq:maximum-gap}  
\max_{\bm w \in \cW} {\rm DG}(\bm v, \bm w) =  
\max_{\bm w \in \cW}  
\left(\boldsymbol{v} \otimes \boldsymbol{w}\right)^{\top} A \left(\boldsymbol{v} \otimes \boldsymbol{w}\right) + \boldsymbol{b}^{\top}\left(\boldsymbol{v} \otimes \boldsymbol{w}\right) + c,  
\end{align}  
where the matrix $ A $ of the quadratic term, the vector $ \bm b $ of the linear term, and the constant term $ c $ are respectively given as  
\begin{align*}  
A  
&=  
\frac{1}{2\lambda} \operatorname{diag}(\bm\alpha_{\bm 1_n, \bm 1_n}^* \otimes \boldsymbol{y})^{\top} K \operatorname{diag}(\bm\alpha_{\bm 1_n, \bm 1_n}^* \otimes \boldsymbol{y}),  
\\  
\bm b  
&=  
\left[\ell(y_i, \bm x_i; \bm \beta_{\bm 1_n, \bm 1_n}^*) - {\alpha}_{\bm 1_n, \bm 1_n, i}^* \right]_{i \in [n]},  
\\  
c  
&=  
\frac{1}{2\lambda} (\bm 1_n \otimes \bm\alpha_{\bm 1_n, \bm 1_n}^* \otimes \boldsymbol{y})^{\top} K (\bm 1_n \otimes \bm\alpha_{\bm 1_n, \bm 1_n}^* \otimes \boldsymbol{y}).  
\end{align*}  

\end{theorem}
%
In what follows, we sketch the proof of Theorem~\ref{theo:main}. The complete proof is given in {Appendix~\ref{app:main-proof}}.


\subsubsection{Proof Sketch}

The proof sketch of Theorem~\ref{theo:main} is primarily divided into the following steps.

\begin{enumerate}
  \item
  First, as a premise, consider a model where $P_{\boldsymbol{w}}$ is $\mu$-strongly convex.
  \item
  Second, under assumption 1, derive the bound of model parameters. The model parameter vector ${\bm \beta}^*_{\bm v, \bm w}$, trained using the coreset vector $\bm v$ and weight vector $\bm w$, is guaranteed to converge within the range ${\cal B}^*_{\bm v, \bm w}$, which is represented by an L2-norm hypersphere \citep{10.1162/neco_a_01619}. 
  %
  A hypersphere of the radius is given by $R = \sqrt{\frac{2}{\lambda}{\rm DG}(\bm v, \bm w)}$, which is calculated in \eqref{eq:duality_gap}.
  %
  \item
  Third, calculate a bound of the weighted validation error.
  %
  Using the hypersphere range ${\cal B}^*_{\bm v, \bm w}$, computed for a specific coreset vector $\bm v$ and weight vector $\bm w$, an upper bound of the weighted validation error can be analytically calculated.
  %
  This upper bound is determined by ${\rm DG}$. The larger ${\rm DG}$, the greater the upper bound becomes.
  %
  \item
  Fourth, maximize ${\rm DG}$ with respect to the weight vector $\bm w$ for training dataset in \eqref{eq:maximum-gap}. We can obtain an upper bound of the worst-case weighted valdation error.
  %
  \item
  Finally, maximize an upper bound of weighted validation error with respect to the weight vector $\bm w^\prime$ for the validation dataset.
  %
  We can derive an upper bound of the worst-case weighted validation error, leading to \eqref{eq:main-bound}.
\end{enumerate}
% 1. 前提条件:
% Primal objective function $P_{\boldsymbol{w}}$が$\mu$-強凸性であるモデルを考える。

% 2. モデルパラメータベクトルの存在範囲
% コアセットベクトルvと重みベクトルwを用いて学習したモデルパラメータベクトル{\bm \beta}^*_{v,w}が、L2ノルムの超球で表される範囲$\mathcal{B}^*_{v,w}$に収束することが保証される\citep{hanada}ことを用いる。範囲$\mathcal{B}^*_{v,w}$の超球半径はR=\sqrt{\frac{2}{\lambda}DG}と表され、v, wについての関数である。

% 2. validation errorの範囲
% 特定のコアセットベクトルvと重みベクトルwに対して計算した超球範囲$\mathcal{B}^*_{v,w}$を用いて、validation errorのupper boundを解析的に計算することができる。この上限は、線形式の値によって定まるものである。すなわち超球半径Rが大きいほど、上限も大きくなる。

% 3. 最悪のケースvalidation error
% validation errorが最悪となることは、超球半径Rの最大化すなわち、DGの最大化と同義である。つまり、wについて最大化を行うことに帰着される。

% 4. 検証データセットの重みベクトルw′が、検証データセットとテストデータセットの入力分布の密度比に基づいて適切に決定されるとする。このとき、式(8)の重み付き検証誤差VaErをw′について最大化すれば、validation errorの上界を定めることができ、式(13)を得る。

% 以上により、最悪ケースのvalidation errorを計算できる。


\subsection{Greedy Coreset Selection Based on the Upper Bound}
\label{subsec:GrCS}
%
The basic idea of the proposed DRCS method is to select the coreset vector $\bm v \in \{0, 1\}^n$ that minimizes the upper bound of the worst-case weighted validation error represented by \eqref{eq:main-bound}.
%
Since this is a combinatorial optimization problem, finding the global optimal solution within a realistic timeframe is challenging; thus, we adopt greedy approaches to obtain approximate solutions.
%

A naive greedy approach involves removing the training instance that minimizes \eqref{eq:main-bound} one at a time (this approach is referred to as {\tt greedy approach 1}).
%
Although {\tt greedy approach 1} is sufficient for small datasets, the computational cost becomes prohibitively high for larger datasets.
%
The most significant computational cost in calculating the bound in \eqref{eq:main-bound} lies in the eigenvalue computation in \eqref{eq:maximum-gap}, required to determine ${\rm max}_{\bm w \in \cW}(\bm v, \bm w)$.
%
To circumvent this cost, another approach, referred to as {\tt greedy approach 2, greedy approach 3}, avoids recomputing the upper bound in \eqref{eq:main-bound} each time an instance is removed.
%
Instead, it selects multiple instances for removal without recomputing the upper bound.
%
These greedy approaches are heuristics and do not guarantee optimality. However, in Section~\ref{sec:experiment}, we demonstrate that these approaches facilitate the selection of a coreset, effectively mitigating the increase in worst-case test error.

The details of these algorithm is given in {Appendix~\ref{app:algorithm}}. In this Appendix, we provide the pseudocode of them.