
\section{Proofs of general lemmas} \label{app:general-lemmas}

In this appendix, we present a collection of lemmas that are utilized in other sections of the appendix.

\begin{lemma}\label{lem:fenchel-moreau} \emph{(Fenchel-Moreau theorem)}
Let $f: \mathbb{R}^d\to\mathbb{R}\cup\{+\infty\}$ be a convex function. The biconjugate $f^{**}$ (\red{$= (f^*)^*$}) coincides with $f$ if $f$ is convex, proper (i.e., $\exists \bm p\in\mathbb{R}^d:~f(\bm p) < +\infty$), and lower-semicontinuous.
\end{lemma}

\begin{proof}
Refer to Section 12 of \citep{rockafellar1970convex} for details.
\end{proof}

As a specific instance of Lemma \ref{lem:fenchel-moreau}, the result holds when $f$ is finite-valued ($\forall \bm p\in\mathbb{R}^d:~f(\bm p)<+\infty$) and convex.

\begin{lemma}\label{lem:strong-convexity-smoothness}
For a convex function $f: \mathbb{R}^d\to\mathbb{R}\cup\{+\infty\}$, the following statements hold:
\begin{itemize}
\item If $f$ is proper and $\nu$-smooth, then $f^*$ is $(1/\nu)$-strongly convex.
\item If $f$ is proper, lower-semicontinuous, and $\kappa$-strongly convex, then $f^*$ is $(1/\kappa)$-smooth.
\end{itemize}
\end{lemma}

\begin{proof}
See Section X.4.2 of \citep{hiriart1993convex} for further explanation.
\end{proof}

\begin{corollary} \label{cor:strong-convexity-smoothness}
Assume that the regularization function $\rho$ in \eqref{eq:primal} is $\kappa$-strongly convex, a condition necessary for applying DRCS. Then, by Lemma \ref{lem:strong-convexity-smoothness}, $\rho^*$ must be $(1/\kappa)$-smooth. This ensures that $\rho^*$ cannot take infinite values in this context.
\end{corollary}

\begin{lemma}\label{lem:strong-convexity-sphere}
Let $f: \mathbb{R}^d\to\mathbb{R}\cup\{+\infty\}$ be a $\kappa$-strongly convex function, and let $\bm p^* = \targmin_{\bm p\in\mathbb{R}^d} f(\bm p)$ be its minimizer. For any $\bm p\in\mathbb{R}^d$, the following inequality holds:
\begin{align*}
\| \bm p - \bm p^* \|_2 \leq \sqrt{\frac{2}{\kappa}[f(\bm p) - f(\bm p^*)]}.
\end{align*}
\end{lemma}

\begin{proof}
Refer to \citep{ndiaye2015gap} for a detailed proof.
\end{proof}

\begin{lemma} \label{lem:optimize-linear}
For any vectors $\bm a, \bm c\in\mathbb{R}^n$ and a positive scalar $S > 0$, the following holds:
\begin{align*}
& \min_{\bm p\in\mathbb{R}^n:~\|\bm p - \bm c\|_2\leq S} \bm a^\top \bm p = \bm a^\top \bm c - S\|\bm a\|_2, \\
& \max_{\bm p\in\mathbb{R}^n:~\|\bm p - \bm c\|_2\leq S} \bm a^\top \bm p = \bm a^\top \bm c + S\|\bm a\|_2.
\end{align*}
\end{lemma}

\begin{proof}
Using the Cauchy-Schwarz inequality, we derive:
\begin{align*}
& -\|\bm a\|_2 \|\bm p - \bm c\|_2 \leq \bm a^\top (\bm p - \bm c) \leq \|\bm a\|_2 \|\bm p - \bm c\|_2.
\end{align*}
The first inequality becomes an equality if $\exists\omega>0:~\bm a = -\omega(\bm p - \bm c)$, and the second inequality becomes an equality if $\exists\omega^\prime>0:~\bm a = \omega^\prime(\bm p - \bm c)$. 

Since $\|\bm p - \bm c\|_2\leq S$, we also have:
\begin{align*}
& - S \|\bm a\|_2 \leq \bm a^\top (\bm p - \bm c) \leq S \|\bm a\|_2,
\end{align*}
with equality when $\|\bm p - \bm c\|_2 = S$.

The optimal $\bm p$ satisfying these conditions is:
\begin{itemize}
\item $\bm p = \bm c - (S/\|\bm a\|_2)\bm a$ for the minimum, and
\item $\bm p = \bm c + (S/\|\bm a\|_2)\bm a$ for the maximum.
\end{itemize}
Thus, the minimum and maximum values of $\bm a^\top (\bm p - \bm c)$ are $- S \|\bm a\|_2$ and $S \|\bm a\|_2$, respectively, completing the proof.
\end{proof}
